\include{settings}

\begin{document}

\begin{titlepage}
\begin{center}
	Санкт-Петербургский Политехнический Университет Петра Великого\\[0.3cm]
	Институт компьютерных наук и технологий \\[0.3cm]
	Кафедра компьютерных систем и программных технологий\\[4cm]
	
	\textbf{ОТЧЕТ}\\ 
	\textbf{по лабораторной работе}\\[0.5cm]
	\textbf{<<Генерация и визуализация исходных данных, \\основы классификации и аппроксимации>>}\\[0.1cm]
	\textbf{Нейроинформатика}\\[4.0cm]
\end{center}

\begin{flushright}
	\begin{minipage}{0.45\textwidth}
		\textbf{Работу выполнил студент}\\[3mm]
		группа 33501/4 \hspace*{10mm} Дьячков В.В.\\[5mm]
		\textbf{Преподаватель}\\[5mm]
		\sign[1.7cm] \hspace*{1mm} к.т.н., доц. Никитин К.В. \\[5mm]
	\end{minipage}
\end{flushright}

\vfill

\begin{center}
	Санкт-Петербург\\
	\the\year
\end{center}
\end{titlepage}

\addtocounter{page}{1}

\section{Цели работы}

\begin{itemize}
\item Научиться формировать выборки, состоящие из обучающих и тестовых примеров
для решения типовых задач классификации, аппроксимации.
\item Овладеть навыками визуализации данных на плоскости при решении задач
классификации и аппроксимации.
\item Научиться рассчитывать основные показатели качества распознавания и
представлять полученные результаты в табличной и графической формах.
\end{itemize}

\section{Крестики-нолики}

\subsection{Задание 1}

\subsection{Задание 2}

\section{Логическая функция 5 переменных}

\subsection{Задание 1}

\subsection{Задание 2}

\section{Разбиение плоскости на 2 класса}

\subsection{Задание 1}

\subsection{Задание 2}

\subsection{Задание 3}

\subsection{Задание 4}

\section{Разбиение плоскости на N классов}

\subsection{Задание 1}

\subsection{Задание 2}

\subsection{Задание 3}

\section{Непрерывная функция одной переменной}

\subsection{Задание 1}

\subsection{Задание 2}

\subsection{Задание 3}

\section{Линейная функция с памятью}

\subsection{Задание 1}

\subsection{Задание 2}

\section{Нелинейная функция с памятью}

\subsection{Задание 1}

\subsection{Задание 2}

\section{Линейное разностное уравнение}

\subsection{Задание 1}

\subsection{Задание 2}

\section{Многомерные образы}

\subsection{Задание 1}

\subsection{Задание 2}

\end{document}